\documentclass{ctuthesis}

\ctusetup{
	xdoctype = M,
	xfaculty = F3,
	mainlanguage = czech,
	title-english = {Scanner/Monitor of IoT radio networks},
	title-czech = {Přehledový přijímač / monitor rádiových sítí IoT},
	department-czech = {Katedra telekomunikační techniky},
	fieldofstudy-czech = {Komunikační systémy a sítě},
	author = {Ondřej Šulc},
	supervisor = {Ing. Pavel Troller, CSc.},
	supervisor-address = {Pestitelský ústav,\\ Zárivá 232,\\12000 Praha 2},
	month = 1,
	year = 2019,
	keywords-czech = {IoT, SDR-RTL, LoRa, Sigfox, Přehledový přijímač},
	keywords-english = {IoT, SDR-RTL, LoRa, Sigfox, Scanner},
	specification-file = {zadani.pdf}
}
\ctuprocess

\begin{abstract-english}
We develop \ldots
\end{abstract-english}

\begin{abstract-czech}
Rozvíjíme \ldots
\end{abstract-czech}

\begin{thanks}
Děkujeme \ldots
\end{thanks}

\begin{declaration}
Fakt sám \ldots
\end{declaration}

\begin{document}



\maketitle

\chapter{Úvod}

Foo bar

\chapter{LoRa}
\section{Modulace}
Modulační schéma LoRa je založeno na Chirp Spread Spread Spectrum (Cvrlikající rozprostřené spektrum) modulaci  (Goursaud and Gorce, 2015) a definuje jeden “cvrk” jako jeden symbol  (Semtech, 2015a). Standardní nemodulovaný lineární cvrk se nazývá “základní cvrk” a může být matematicky popsán jako funkce času t takto (Mann and Haykin, 1991):
\begin{align}x(t)=e^{i(\varphi_{0}+2\pi(\frac{k}{2}t^{2} + f_{0}t))}
\label{eq:mod1}
\end{align}
Kde  $\varphi_{0}$ je počáteční fáze, $k$ je rychlost změny frekvence a $f_{0}$ je počáteční frekvence. Pokud je šířka pásma kanálu $BW$, tak parametry $f_{0}$ a $k$ jsou nastaveny tak, že se frekvence zvětšuje od $f_{0}-\frac{BW}{2}$ po $f_{0}+\frac{BW}{2}$ během periody $T$ cvrku. Tím pádem je $f_{0}=\frac{BW}{2}$ and $k = \frac{BW}{T}$. Doba trvání jednoho cvrku závisí na šířce pásma signálu a na parametru nazývaném činitel rozprostření (Spreading Factor - SF) dle vztahu $T = \frac{2^{SF}}{BW}$ (Seller and Sornin, 2014).
Vzhledem k tomu, že $x(t + nT) = x(t)$ kde $n\in \mathbb{N}$, celočíselná hodnota $i \in \{0, 1\}^{SF}$ může být namodulována na základní cvrk pomocí časového posunu $\hat{t} = Gray^{-1}(i)\frac{T}{2^{SF}}$ aplikovaného na signál ve vztahu \eqref{eq:mod1},  kde $Gray^{1}$ je dekódování Grayova kódu (Gray, 1953). Touto cestou je symbol v podstatě kvantovaný na $2^{SF}$ časových intervalů rozdělujích šířku pásma, nazýváme je “chipy” a právě ony určují $i$. Při příjmu modulovaného cvrku s neznámým časovým posuvem $x(t + \hat{t})$, může být hodnota cvrku zrekonstruována navzorkováním signálu vzorkovací frekvencí chipů a výpočtem:
\begin{align}i= Gray(arg \max (\lvert FFT(x(t+ \hat{t}) \odot \overline{x(t)}) \rvert ))
\label{eq:mod2}
\end{align}
Kde $\overline{x(t)}$ značí komplexně sdružený základní cvrk, $\odot$ značí multiplikaci po prvcích, $\lvert FFT(x) \rvert$ zančí velikost Rychlé Fourierovi transformace $x$, a $Gray$ je Grayovo kódování. 

\chapter{Závěr}

Lorep ipsum \cite{doe}

\begin{thebibliography}{1}

\bibitem{doe} J. Doe. \emph{Book on foobar.} Publisher X,
 2300.

\end{thebibliography}

\end{document}
