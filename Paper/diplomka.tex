\documentclass{ctuthesis}

\ctusetup{
	xdoctype = M,
	xfaculty = F3,
	mainlanguage = czech,
	title-english = {Scanner/Monitor of IoT radio networks},
	title-czech = {Přehledový přijímač / monitor rádiových sítí IoT},
	department-czech = {Katedra telekomunikační techniky},
	fieldofstudy-czech = {Komunikační systémy a sítě},
	author = {Ondřej Šulc},
	supervisor = {Ing. Pavel Troller, CSc.},
	supervisor-address = {Pestitelský ústav,\\ Zárivá 232,\\12000 Praha 2},
	month = 1,
	year = 2019,
	keywords-czech = {IoT, SDR-RTL, LoRa, Sigfox, Přehledový přijímač},
	keywords-english = {IoT, SDR-RTL, LoRa, Sigfox, Scanner},
	specification-file = {zadani.pdf}
}
\ctuprocess

\begin{abstract-english}
We develop \ldots
\end{abstract-english}

\begin{abstract-czech}
Rozvíjíme \ldots
\end{abstract-czech}

\begin{thanks}
Děkujeme \ldots
\end{thanks}

\begin{declaration}
Fakt sám \ldots
\end{declaration}

\begin{document}



\maketitle

\chapter{Úvod}

Foo bar

\chapter{LoRa}
\section{Fyzická vrstva (LoRa PHY)}
\subsection{Modulace}
Modulační schéma LoRa je založeno na Chirp Spread Spread Spectrum (Cvrlikající rozprostřené spektrum) modulaci  (Goursaud and Gorce, 2015) a definuje jeden “cvrk” jako jeden symbol  (Semtech, 2015a). Standardní nemodulovaný lineární cvrk se nazývá “základní cvrk” a může být matematicky popsán jako funkce času t takto (Mann and Haykin, 1991):
\begin{align}x(t)=e^{i(\varphi_{0}+2\pi(\frac{k}{2}t^{2} + f_{0}t))}
\label{eq:mod1}
\end{align}
Kde  $\varphi_{0}$ je počáteční fáze, $k$ je rychlost změny frekvence a $f_{0}$ je počáteční frekvence. Pokud je šířka pásma kanálu $BW$, tak parametry $f_{0}$ a $k$ jsou nastaveny tak, že se frekvence zvětšuje od $f_{0}-\frac{BW}{2}$ po $f_{0}+\frac{BW}{2}$ během periody $T$ cvrku. Tím pádem je $f_{0}=\frac{BW}{2}$ and $k = \frac{BW}{T}$. Doba trvání jednoho cvrku závisí na šířce pásma signálu a na parametru nazývaném činitel rozprostření (Spreading Factor - SF) dle vztahu $T = \frac{2^{SF}}{BW}$ (Seller and Sornin, 2014).
Vzhledem k tomu, že $x(t + nT) = x(t)$ kde $n\in \mathbb{N}$, celočíselná hodnota $i \in \{0, 1\}^{SF}$ může být namodulována na základní cvrk pomocí časového posunu $\hat{t} = Gray^{-1}(i)\frac{T}{2^{SF}}$ aplikovaného na signál ve vztahu \eqref{eq:mod1},  kde $Gray^{1}$ je dekódování Grayova kódu (Gray, 1953). Touto cestou je symbol v podstatě kvantovaný na $2^{SF}$ časových intervalů rozdělujích šířku pásma, nazýváme je “chipy” a právě ony určují $i$. Při příjmu modulovaného cvrku s neznámým časovým posuvem $x(t + \hat{t})$, může být hodnota cvrku zrekonstruována navzorkováním signálu vzorkovací frekvencí chipů a výpočtem:
\begin{align}i= Gray(arg \max (\lvert FFT(x(t+ \hat{t}) \odot \overline{x(t)}) \rvert ))
\label{eq:mod2}
\end{align}
Kde $\overline{x(t)}$ značí komplexně sdružený základní cvrk, $\odot$ značí multiplikaci po prvcích, $\lvert FFT(x) \rvert$ zančí velikost Rychlé Fourierovi transformace $x$, a $Gray$ je Grayovo kódování. 

\subsection{Prokládání}
Jako v každé jiné modulaci, musíme i zde počítat s chybami způsobenými šumem, interferencí, a časovými nebo frekvenčními posuny. Tyto chyby mohou způsobit, že hodnota čipu nebude dobře odečtena z modulovaného symbolu. Například poryv šumu může posunout vrchol v FFT spektru na jinou hodnotu chipu a tak jej znehodnotit.\\
Aby bylo možné minimalizovat dopad poryvů šumu na chybu jen jednoho bitu v symbolu je použito prokládání. Několik chipů je dohromady vepsáno do mřížky $\{0,1\}^{SF x (4 + CR)}$, kde CR (Coding Rate) značí počet paritních bitů a nabývá hodnot 1 až 4. Pokud tedy bude použit $SF = 7$ a $CR =4$ dostaneme matici  $\{0,1\}^{7 x 8}$, příklad je na obrázku \ref{fig:mod1}. K sískání kódové slova je pak potřeba číst bity po diagonále matice. Na rozdíl od patentu LoRy (Seller and Sornin, 2014), kde se uvádí, že směr diagonálního čtení bitů z mřížky je směrem dolů, v praxi lze pozorovat opačný směr. Tímto způsobem tak první chip obsahuje všechny nejméně významné bity (LSB - Least significant) všech kódových slov, druhý čip všechny druhé bity všech slov a tak dále. Díky tomu v případě ztráty celého čipu dojde k chybě jen v jednom bitu na kódové slovo.\\
Dalším způsobem jak zvýšit odolnost proti rušení vysílání je použití módu redukované rychlosti (reduced rate mode). V případě použití tohoto módu jsou první dvě řady prokládací matice zahozeny a její rozměr se tak změní na $\{0,1\}^{SF-2 x (4 + CR)}$ což způsobí, že z ní nasledně vyčteme o dvě kódová slova méně. Zahozené řádky obsahují nejméně významné bity chipů, které jsou nachylnější k chybám protože odpovídají užším frekvenčním intervalům v FFT spektru. Z toho vyplývá, že mód redukované rychlosti obětuje rychlost přenosu dat ve prospěch odolnosti proti šumu. Hlavička fyzické vrstvy LoRa je v tomto módu vysílána vždy, kdežtkoo užitečná data je v případě použítí SF 11 nebo 12.

\subsection{Kódování}
Po přečtení kódových slov z prokládací matice mají tato délku $4 + CR$. Kvůli zamezení vzniku stejnosměrné složky byla slova v části rámce s užitečnými daty XOR-ována 9-bitovým lineární posuvným registrem se zpětnou vazbou (LSFR Linear feedback shift register) (whitening). A proto musí po synchronizaci projít stejným procesem znovu. Přesný algoritmus není v patentu určen a jeho výběr je tedy na každém výrobci zvlášť. \\
Na několika testovacích zařízeních \ref{nejdulzittejsipapir} reverzním inženýrstvým zjistilo použité upraveného $4/(4 + CR)$ Hammingova kódu. Ve výsledku tak z každého kódového slova po dekódování získáme 4 bity dat. Ta jsou pak naparsována du struktury rámce lora.

\subsection{Struktura rámce}
Na fyzické vrstě LoRa definuje rámec jako strukturu složenou z následujících polí. Pole jsou uvedena ve stejném pořadí jako v rámci.  (Semtech, 2015b, p. 27–29)
\begin{description}
\item[Preambule]
Sekvence základních cvrků, která slouží k časové a frekvenční synchronizaci. Počet cvrků není pevně dán.
\item[Symboly synchronizace rámce]
Dva modulované cvrky co mouhou být použity pro identifikaci sítě. Hardwarový přijímač zahodí rámcec, které obsahují synchronizační symboly co neodpovídají jeho nastavení.
\item[Symboly synchronizace frekvence]
Dva sdružené cvrky následované sdruženým cvrkem s periodoou $\frac{T}{4}$ určené pro přesnou frekvenční synchronizaci.
\item[Hlavička (nepoviná)]
Hlavička obsahuje délku užitečných dat, použitou přenosovou rychlost, indikuje použití Cyklického redundantního součtu (CRC - Cyclic redunduncy check) a jendobajtovou kontrolní sumu hlavičky. Pro modulaci hlavičky je vždy použito $CR =4$ a mód redukované rychlosti. Pokud hlavička vysílána není (implicitní mód) musí mít jak přijímač tak vysílač předem schodně nastavený CR a také zdali je použito CRC.
\item[Užitečná data]
Pole o proměnné délce obsahující data vrstvy přístupu k médiu (MAC - Media access control) a případné dvoubajtové CRC těchto dat.
\end{description}
\subsection{Struktura hlavičky}
Délka hlavičky není ve specifikaci nikdy přímo určena. Lze jí však vydedukovat z toho, že hlavička je vždy vysílána v módu redukované rychlosti, má $CR =4$ a SF minimálně 7. Z toho vyplívá že hlavička se musí vejít do mřížky $\{0,1\}^{7-2 x 8)}$ a to odpovídá 5 kódovým slovům. Každé slovo má 8 bitů a dohromady je to bitů 40. Jakékoliv zbývající bity jsou použity pro užitečná data. \\
Po dekódování díky redundantním bitům dostáváme $40\frac{4}{8} = 20$ bitů nebo 2,5 bajtu. V \ref{nejdulezitejsipapir} experimentálně vyzkoušeli pořadí hlavičky. První bajt udává délku datového obsahu, následuje půlslabika  udávající CR a přítomnost MAC CRC a poslední bajt obsahuje kontrolní součet hlavičky, z něj je však používá jen 5 LSB bitů.


\chapter{Závěr}

Lorep ipsum \cite{doe}

\begin{thebibliography}{1}

\bibitem{doe} J. Doe. \emph{Book on foobar.} Publisher X,
 2300.

\end{thebibliography}

\end{document}
